\newif\ifrevtex
\revtextrue

\ifrevtex
    \documentclass[aps,final,twocolumn,letterpaper,nofootinbib]{revtex4-1}
\else
    \documentclass[twocolumn]{article}
\fi

%\usepackage[letterpaper,hmargin=0.8in,vmargin=0.8in]{geometry}
\usepackage{graphicx}

\usepackage{amsmath,amssymb}
\usepackage{siunitx}

%\usepackage{mathpazo}
\usepackage{microtype}

\usepackage{diagbox}

\usepackage{color}
\usepackage[dvipsnames]{xcolor}
\usepackage[printwatermark]{xwatermark}
\newwatermark[allpages,color=red!10,angle=45,scale=3,xpos=0,ypos=0]{DRAFT}

\newcommand\RR{\mathbb{R}}

%\usepackage{tikz}
%\usetikzlibrary{arrows}
%\usetikzlibrary{angles,patterns,calc}

\newcommand\headers{
    \title{Rigiditea: web-based rigidity algorithms}
    \author{Tony Zhang and Menghua Wu}
    \date{May 17, 2017}    
    \begin{abstract}
        We present Rigiditea,
        a native client-side web-based implementation of two algorithms
        for testing generic rigidty of graphs:
        the $n$-dimensional randomized infinitesimal-rigidity-based algorithm
        and the deterministic pebble game algorithm for two-dimensional rigidity.
        For the latter, we provide a step-by-step visualization of the algorithm
        for aid in understanding the algorithm.
    \end{abstract}
}


\ifrevtex\relax\else\headers\fi
\begin{document}
\ifrevtex\headers\fi

\maketitle



% % % % % % % % % %
%    INTRODUCTION
% % % % % % % % % %

\tableofcontents

\section{Introduction}

The theory of linkages TODO TODO TODO

We implemented two major algorithms for generic rigidity.
The first, which works in $n$ dimensions,
considers a random embedding of the graph into $\RR^n$
and computes the number of infinitesimal degrees of freedom (dof).
Infinitesimal rigidity of the random embedding always implies generic rigidity;
the converse holds with high probability.

The second algorithm is the famous \emph{pebble game algorithm},
first introduced by Jacobs and Hendrickson in 1997~\cite{jacobs97}.
Briefly, it considers TODO TOOD

To the best of our knowledge,
there do not exist convenient implementations of
either rigidity algorithms.
In particular,
we were unable to find an sort of graphical interface
for the infinitesimal algorithm,
and only found a Java applet for the pebble game \cite{stjohnapplet}.


\section{Interface}

\section{Pebble game visualization}

\section{Infinitesimal rigidity}

\section{Discussion}

\section{Extensions}

\begin{itemize}
    \item
    Undo feature
    \item
    Graph export into FOLD
    \item
    Visualize rigid components
    \item
    Extend to other pebble games like in \cite{lee08}
\end{itemize}


\section*{Acknowledgements}


\bibliography{rigiditea}{}
\bibliographystyle{plain}






\end{document}

