\newif\ifrevtex
\revtextrue

\ifrevtex
    \documentclass[aps,prd,final,twocolumn,letterpaper,nofootinbib]{revtex4-1}
\else
    \documentclass[twocolumn]{article}
\fi

%\usepackage[letterpaper,hmargin=0.8in,vmargin=0.8in]{geometry}
\usepackage{graphicx}

\usepackage{amsmath,amssymb}
\usepackage{siunitx}


%\usepackage{mathpazo}
\usepackage{microtype}

\usepackage{diagbox}
\usepackage{multirow}

\usepackage{color}
%\usepackage[printwatermark]{xwatermark}
%\newwatermark[allpages,color=red!10,angle=45,scale=3,xpos=0,ypos=0]{DRAFT}

\usepackage{tikz}

\usepackage{hyperref}
\usepackage{cleveref}



\newcommand\RR{\mathbb{R}}
\DeclareMathOperator\im{im}

\newcommand{\abs}[1]{|#1|}


\newcommand{\mb}{\mathbf}

%\usepackage{tikz}
%\usetikzlibrary{arrows}
%\usetikzlibrary{angles,patterns,calc}

\newcommand\headers{
    \title{Rigiditea: web-based rigidity algorithms}
    \author{Tony Zhang and Menghua Wu}
    \date{May 17, 2017}    
    \begin{abstract}
        We present Rigiditea,
        a native client-side web-based implementation of two algorithms
        for testing generic rigidity of graphs:
        the $n$-dimensional randomized infinitesimal-rigidity-based algorithm
        and the deterministic pebble game algorithm for two-dimensional rigidity.
        For the latter, we provide a step-by-step visualization of the algorithm
        for aid in understanding the algorithm.
    \end{abstract}
}


\ifrevtex\relax\else\headers\fi
\begin{document}
\ifrevtex\headers\fi

\maketitle



% % % % % % % % % %
%    INTRODUCTION
% % % % % % % % % %

\tableofcontents

\section{Introduction}

The theory of linkages has long been of interest
not only for its mathematical appeal
but also for its many obvious applications
(e.g. modeling robots, proteins, trusses).
An obvious question is that of \emph{rigidity}:
whether a linkage can move at all.

It's not hard to see that rigidity does not depend
only on the combinatorial structure of the linkage
(usually specified as an undirected graph).
Indeed, the number of dimensions in which we embed the linkage
and the particular configuration changes the rigidity.

Fortunately, if we only consider \emph{generic} embeddings
(which we take for simplicity to mean
there are no algebraic dependence between coordinates),
rigidity (and more generally, the number of degrees of freedom)
becomes a function only of the graph and the embedding dimension.

We implemented two major algorithms for generic rigidity
as well as a web-based interface for using them.
The first algorithm, which works in $d$ dimensions,
considers a random embedding of the graph into $\RR^n$
and computes the number of \emph{infinitesimal} degrees of freedom (dof)
as described in \cite[\S4.4.2]{gfalop}.
The second is the famous \emph{pebble game algorithm},
first introduced by Jacobs and Hendrickson in 1997~\cite{jacobs97}.

To the best of our knowledge,
there did not exist convenient implementations of
either rigidity algorithm.
In particular,
we were unable to find any implementation of the infinitesimal algorithm,
and only found a Java applet for the pebble game \cite{stjohnapplet}.

Let us outline the remainder of this paper.
In \cref{sec:arch}, we describe the structure of our web application.
We next develop the theory behind the two rigidity algorithms
in \cref{sec:pebble,sec:infrigid}
to the level of detail necessary for understanding our work.
(Notably, we introduce a correction in \cref{sec:infrigid}
to the well-known formula for the infinitesimal dof of a linkage.)
We conclude with a comparison of to the existing Java applet
and a discussion of possible extensions to Rigiditea in \cref{sec:discuss}.

\section{Architecture}
\label{sec:arch}

We provide a high-level overview of our web application
and algorithms to acquaint users with Rigiditea.
All of our code was written in CoffeeScript,
which was transpiled down to JavaScript.
Thus, our entire app runs in the browser.
Still, there was a clear divide between user interface code
and ``backend", algorithmic code.

\subsection{Interface}

We designed an aesthetic graphical user interface (GUI)
to create and analyze linkages as 2D planar graphs.

\begin{figure}[ht]
\begin{tikzpicture}
\node at (0,0) {
\includegraphics[scale=.3]{img/ui}
};
\draw (-3,-2.7) rectangle (3,2);
\node at (0,-1.5) {\large graph-drawing canvas};

\draw (-2,2.2) rectangle (2,2.7);
\node at (-2.5,2.45) {menu};

\draw (-1.5,-3) rectangle (1.5,-3.3);
\node at (-2.4,-3.15) {controls};
\end{tikzpicture}
\caption{High level overview of graphical user interface.
We aimed for a simple, but powerful design.}
\label{fig:ui}
\end{figure}

First, a user may click anywhere on the canvas to add a new node to the graph.
After creating nodes,
a user can control the user interface through one of two ways:
the menu bar and the control panel.
The former focuses on algorithm running and general information,
while the latter is tailored towards graph creation.
\Cref{fig:ui} points out these elements in the actual interface.
In addition to the menu and control panel,
we also provide keyboard shortcuts to streamline the user experience.
\Cref{tab:menu} enumerates the various commands available,
their functionalities, and keyboard shortcuts.

\begin{table}[ht]
\caption{Control panel items and descriptions.}
\def\arraystretch{1.5}
\begin{tabular}{c | c | p{0.5\linewidth} | c}
& item & description & key \\ \hline
\multirow{3}{.5cm}{\rotatebox{90}{menu bar items~~}}
& pebble & advances pebble algorithm one step
(one cover enlargement; see \cref{sec:pebble}) & $\to$\\
& reset & resets defaults, initializes new graph, wipes canvas & r \\
& about & background information about algorithms,
web application, and further reading & a \\\hline
\multirow{3}{.5cm}{\rotatebox{90}{control panel~~}} &
edge & draws an edge between two selected vertices & e\\
& delete & deletes all selected components, including incident edges
to selected nodes & x \\
& help & provides information about graph designing interface,
including keyboard shortcuts & h \\
\end{tabular}
\label{tab:menu}
\end{table}

The GUI was created using \texttt{d3.js},
an open-source data visualization library for web applications.
We selected this library
to directly link graph data objects to SVG elements on the canvas.

\subsection{Graph representation}

Graphs were represented as custom Javascript \texttt{Graph} objects,
each with a list of \texttt{nodes}, \texttt{edges}, and specialized instance methods.
We will later introduce the \texttt{PebbleGraph} class,
which is a subclass of \texttt{Graph} with additional rigidity-related capabilities.

Each \texttt{Edge} object keeps track of its unique \texttt{id}, \texttt{source}, \texttt{target}, and \texttt{attributes},
such as current and previous color.
Of these fields, \texttt{source} and \texttt{target} are both \texttt{Node} objects,
which each possess a unique \texttt{id}, its \texttt{x} and \texttt{y} coordinates,
and a map of \texttt{attributes}.
Unique \texttt{id}s were randomly generated,
case-sensitive alphanumeric strings of 5 characters.
\Cref{fig:graphrep} provides a simple example.

Note that nodes are passed to \texttt{edges} by reference.
Furthermore, we use an absolute coordinate system,
based on pixels, so for convenience, each node keeps track of its own location.

\begin{figure}[ht]
\begin{verbatim}
Graph: {
   nodes: {
     Node:
       {id: a6849, x:68, y:49, attr: {fill: #CEF}},
     Node:
       {id: b6867, x:68, y:67, attr: {fill: #FCE}}},
   edges: {
     Edge:
       {source: Node, target: Node,
         attr: {stroke: #CEF}}},
         ...
   [instance methods]
}
\end{verbatim}
\caption{Sample representation of a line segment.}
\label{fig:graphrep}
\end{figure}

On top of this basic graph,
we designed another \texttt{PebbleGraph} class,
which could run iterations of the pebble algorithm
and test for infinitesimal rigidity.
\texttt{PebbleGraph} objects can be easily created
from standard \texttt{Graph} objects since
their underlying representations are the same.
However, \texttt{PebbleGraph} objects also maintain
an assignment of ``pebbles'' to edges and vertices.
Thus, once we begin running the Pebble algorithm,
we disable graph editing in the GUI and generate a mutable \texttt{PebbleGraph}
for future steps.

\section{Pebble game visualization}
\label{sec:pebble}

We shall sketch the pebble game algorithm
for generic two-dimensional rigidity testing
and describe our visualization for it.
For a more detailed exposition on the algorithm,
we refer the reader to the original paper~\cite{jacobs97}.

\subsection{The algorithm}

Before describing the algorithm,
we present some basic definitions.
Consider undirected graph $G = (V, E)$
with $n$ vertices and $m$ edges.
We say a subset of the edges is \emph{independent}
if they produce independent constraints on the motion of the vertices
in a generic embedding.
An independent set is maximal if we can't add another edge
while preserving independence.

In two dimensions, $n$ unconstrained vertices have $2n-3$ degrees of freedom,
where we have excluded rigid motion.
Now each independent edge reduces this freedom by one,
so that
\[
    \text{degrees of freedom} = 2n - 3 - \abs{\text{maximal independent set}}.
\]
It follows that our graph is generically rigid
if we can find $2n-3$ independent edges.\footnote{It makes sense
to talk about \emph{the} size 
of maximal independent sets
for the same reason that the dimension of a span of vectors
is independent of the order in which we consider them;
indeed, edges correspond to vectors in $\RR^{dn}$,
as we shall see in the next section.}

On a high level,
the pebble algorithm builds an independent set of edges
by arbitrarily ordering the edges
and accepting an edge into the current independent set
if and only if doing so would preserve independence.
After all edges are considered,
we have a maximal independent set,
so testing rigidity becomes a simple matter.

The key insight of the algorithm
is the observation that we can test independence of a new edge
and a known independent set
by playing the namesake pebble game.
We shall largely follow \cite{stjohnapplet}.

Each vertex is given two pebbles it can use to cover incident edges.
At any given moment, we are attempting to add some edge $e$
to the�partially-built independent set $\hat E$.
We already have a pebber covering such that all edges in $\hat E$
are covered by at least one pebble.

To test independence of $e$,
it actually suffices to check whether we can \emph{enlarge} the covering of $\hat E$
to a covering of $\hat E \cup \{e\}$
such that $e$ is covered by \emph{four} pebbles,
while keeping all of $\hat E$ covered.

To find such a covering,
we define the following enlargement subroutine,
which tries to get an extra pebble to cover $e$.
Starting at one of the endpoints of $e$,
we look for a vertex or an edge with a free pebble
(that is, a pebble not covering an edge
or one redundantly covering an edge)
via depth-first search.
Should we find a free pebble,
we rearrange pebbles along the path so that $e$ has an extra pebble.
Otherwise, we attempt the same operation on the other endpoint of $e$.

If we cannot successfully apply this subroutine four times to $e$,
we discard $e$ from consideration.
Otherwise, we add it to $\hat E$ and consider a new $e$.

This algorithm runs in $O(mn)$ time:
we have $O(m)$ iterations,
each of which involves at most four $O(n)$ depth-first searches.
In practice, we did not bother with making our implementation particularly efficient,
since the performance bottleneck was the rendering code.


\subsection{Visualization}

\begin{figure}[h]
   \centering
   \includegraphics[width=.19\linewidth]{img/l1}
   \includegraphics[width=.19\linewidth]{img/l2}
   \includegraphics[width=.19\linewidth]{img/l3}
   \includegraphics[width=.19\linewidth]{img/l4}
   \includegraphics[width=.19\linewidth]{img/l5}
   \caption{Pebble cover on a single segment.
   We start by drawing the segment
   and proceed to cover it with pebbles,
   until determining that a segment is rigid.}
   \label{fig:seg}
\end{figure}

Rigiditea primarily provides a pebble game visualization,
in which we attempt to add edges and see if they contribute
to the graph's overall rigidity.

At each phase, we randomly select an edge to ``add'' to our current set.
For four rounds, we attempt to cover the edge with four pebbles.
This is reflected in our GUI with a spectrum of blue and green,
where zero pebbles is a darker blue and four pebbles is a light yellow-green.
Pebbles can reside in both nodes and edges,
so both can be colored.
Figure \ref{fig:seg} shows a simplified version for a single segment,
and figure \ref{fig:pebbles} demonstrates this process
for a triangle.

At any given point during the pebble game,
a user can view the number of pebbles on an edge or node by
mousing over that element.

\begin{figure*}[t]
   \centering
   \includegraphics[width=.24\linewidth]{img/t1}
   \includegraphics[width=.24\linewidth]{img/t2}
   \includegraphics[width=.24\linewidth]{img/t3}
   \includegraphics[width=.24\linewidth]{img/t4}
   \includegraphics[width=.24\linewidth]{img/t5}
   \includegraphics[width=.24\linewidth]{img/t6}
   \includegraphics[width=.24\linewidth]{img/t7}
   \includegraphics[width=.24\linewidth]{img/t8}
   \includegraphics[width=.24\linewidth]{img/t9}
   \includegraphics[width=.24\linewidth]{img/t10}
   \includegraphics[width=.24\linewidth]{img/t11}
   \includegraphics[width=.24\linewidth]{img/t12}
   \caption{Process of a single pebble game for a triangle.
   The labels on each edge or node show the number of pebbles on that element.
   We can see that the number of pebbles increases each time from 1 to 4,
   as the pebbles on nodes are transferred to edges.
   In the final frame, we see that the triangle is successfully identified as rigid.}
   \label{fig:pebbles}
\end{figure*}

\section{Infinitesimal rigidity}
\label{sec:infrigid}

In this section,
we outline the algorithm for determining the infinitesimal degrees of freedom
of a $d$-dimensional linkage embedding.
The algorithm is straightforwardly presented in \cite[\S4.4.2]{gfalop},
so we only provide a brief sketch.
More importantly, however,
we shall prove a correction to the formula the algorithm uses
to compute the infinitesimal degrees of freedom.

As before, suppose we have $G = (V, E)$ of $n$ vertices and $m$ edges.
Given a $d$-dimensional embedding of the vertices,
each edge provides a linear constraint
on the allowed infinitesimal motions of $V$.
In matrix form, we can write
\begin{equation}
    R\mb v = 0
\end{equation}
for a \emph{rigidity matrix}~$R$ of shape $dn \times m$
and an infinitesimal motion $\mb v \in \RR^{dn}$,
whose components are the $d$ components
of the infinitesimal motion of each of $n$ vertices.

The space of allowed infinitesimal motions is then $\ker R$.
However, some of these motions are rigid,
corresponding to the infinitesimal generators of rigid motions in $\RR^d$,
which form a $\binom{d+1}{2}$-dimensional vector space:
the Lie algebra $\mathfrak{se}(d)$ of the \emph{special Euclidean group},
the orientation-preserving isometry group of $\RR^d$.

Once we exclude these trivial motions,
we would naively expect the subspace of nontrivial infinitesimal motions
to have dimension
\begin{equation}\label{eq:wrong-infdof}
    \dim\ker R - \binom{d+1}{2} = dn - \dim\im R - \binom{d+1}{2},
\end{equation}
as presented in \cite{gfalop}.

This formula is generally correct,
but fails badly for some elementary examples.
For instance, take $K_2$, the graph of two vertices joined by an edge.
The rigidity matrix will have rank 1,
so for sufficiently large $d$,
we expect negative degrees of freedom:
\[
    dn - 1 - \underbrace{\binom{d+1}{2}}_{\Theta(d^2)} < 0.
\]
Of course, $K_2$ should have no degrees of freedom for any~$d$,
so we're clearly overcompensating in accounting for rigid motions.

Indeed, the naive analysis above
assumed that the subspace~$M_0$ of trivial motions in $\ker R$
was $\binom{d+1}{2}$-dimensional:
that is, each nontrivial infinitesimal isometry of $\RR^d$
produced a nontrivial infintiesimal motion of our graph.
This assumption is generally false.
In our $K_2$ example,
if $d=3$, 
the infinitesimal generator
of the rotation about the single edge
produces the trivial infinitesimal motion 0
of the two vertices.

We'd like to find $\dim M_0$,
which will give the correct number of independent rigid motions
for which we need to compensate in \cref{eq:wrong-infdof}.
To this end,
consider the linear map $\phi\colon\mathfrak{se}(d) \to M_0$
that takes an infinitesimal Euclidean isometry
and gives the associated rigid motion of the linkage.

The dimension formula gives
\[
    \dim M_0 = \dim\mathfrak{se}(d) - \dim\ker\phi.
\]
The first term is the well-known dimensionality $\binom{d+1}{2}$
of rigid motions in $\RR^d$.
The second term takes more care.

By definition, $\ker\phi$ gives the infinitesimal isometries
that fix the vertices of the linkage,
Let the origin of $\RR^d$ be one of these vertices
and note that any $u\in \mathfrak{se}(d)$ fixes the origin;
therefore, $u\in\mathfrak{so}(d)$
(the Lie algebra of the group
of orientation-preserving orthogonal linear maps on $\RR^d$)
is a linear operator on $\RR^d$.

Since $u$ generates an isometry fixing all vertices of the linkage,
$u$ vanishes on all of them.
So $u$ vanishes on the lowest-dimensional flat containing all the vertices;
say this flat has dimension $k$.
Then $\ker\phi\cong\mathfrak{so}(d-k)$, which has dimension $\binom{d-k}{2}$.
(Intuitively: if a $d$-dimensional rotation fixes a $k$-dimensional subspace,
it's really a $d-k$-dimensional rotation.)

As a result, the correct formula we want is
\begin{equation}
    dn - \dim\im R - \binom{d+1}{2} + \binom{d-k}{2}.
\end{equation}
We are not aware of prior work mentioning the final correction term we derived.

Note that this algorithm immediately turns into an algorithm
for testing generic linkage rigidity in arbitrarily many dimensions $d$.
Since random embeddings are generic with high probability,
the infinitesimal dof will almost always match the generic dof.
So testing generic rigidity reduces to testing infinitesimal rigidity.


\section{Discussion}
\label{sec:discuss}

\subsection{Comparison with existing applet}


\begin{table*}[ht]
\def\arraystretch{1.5}
\caption{Comparison of Rigiditea and existing Java implementation
across different features and criteria.}
\begin{tabular}{p{0.4\linewidth} | p{0.4\linewidth}}
Rigiditea & Java implementation \\ \hline
minimalist web GUI, completely front-end, can be hosted anywhere
without a backend &
Java application, requires download to run (most systems have a JVM) \\
aesthetic design, fun to play with, keyboard shortcuts convenient &
standard software developer look, though with intuitive icons
and a thorough menu\\
draw graphs completely freehand,
with simple editing tools &
select from a set of sample graphs,
which demonstrate certain concepts \\
graphs are fixed in location and scale &
graphs can be rotated and moved around point or segment \\
visualize individual steps of pebble algorithm &
visualize individual steps of pebble algorithm\\
tests for linkage rigidity & tests for linkage rigidity\\
only provides overall rigidity & finds over-braced components \\
provides tester for infinitesimal rigidity in $n$ dimensions,
independent of the pebble algorithm
& not a design goal
\end{tabular}
\label{tab:comp}
\end{table*}


Rigiditea is not the first application to visualize linkage rigidity.
There currently exists another
implementation of the pebble algorithm as a Java application~\cite{stjohnapplet}.

Compared to Rigiditea, this implementation
has more built-in features and is a more mature application,
but as a fully front-end web application,
Rigiditea is more accessible, easy-to-use, and pleasing to look at.
Rigiditea also offers complete flexibility in graph design,
instead of using several existing shapes for demonstration purposes.
\Cref{tab:comp} provides an extensive comparison of the two implementations.
On a whole, Rigiditea is not meant to replace the existing tool,
but rather to provide a simpler tool for those casually interested
in rigidity.

Moreover, we envision Rigiditea as a means for laypersons
to learn more about rigidity in general.
Towards these aspects, we should perhaps add additional features,
but due to time constraints, we opted for just a rigidity tester.

\subsection{Future work}

While Rigiditea is currently a simple, functional web application,
there are several extensions we may explore for Rigiditea
to improve its functionality and user-friendliness.

\subsubsection{Web interface}

It would be convenient to scroll through the pebble algorithm's steps
in a timeline, similar to that of the existing Java implementation.
This feature could be easy to implement,
if we simply maintain a history of the \texttt{PebbleGraph} objects
and display one of them depending on the time step.
Afterwards, we could optimize this history by storing a log
of updates and changes to the \texttt{PebbleGraph},
rather than replicating the object $t$ times for $t$ time steps.

Furthermore, there are several GUI features that would be nice to have,
which we ran out of time to implement.
To improve our graph editing capabilities,
we should include the ability to drag nodes around with \texttt{ALT+click}.
It would also be ideal to snap nodes to a grid,
so the overall design looks neater.

Finally, it would be convenient to export the resultant graph into FOLD,
as a linkage. However, we did not prioritize this goal since
linkages can be adequately visualized as planar graphs,
which are quite conceptually general.
We also did not envision this application as a graph-sharing website.
FOLD may also present alternative means of visualizing the linkage,
though we are quite satisfied with our current visualization.

\subsubsection{Algorithmic extensions}

Algorithmically, we would also like to visualize rigid components of a linkage.
It would be ideal to color the over-braced,
under-braced, and minimally rigid components so a user
can distinguish between them.
The existing Java implementation shows rigid components,
though it only applies the algorithm to a few pre-determined linkages.
The pebble algorithm can be used to determine such components,
but we did not have the time to implement that element here.

Finally, we would like to extend this web application to other pebble games
such as those presented in \cite{lee08, chubynsky07}.
% TODO TODO TODO TODO describe these more
TODO TODOTODO

Since we have already built a web interface for visualizing pebble games,
it would be a shame not to use it.


\appendix*
\section{Miscellaneous remarks}

\subsection{Challenges}

We encountered several challenges along the way.
First, Tony didn't realize his CoffeeScript didn't compile---until the ``morning''
of the presentation.
Second, Rachel didn't realize that a button would be easier to use
and infinitely easier to implement than click-and-drag.

In seriousness, both of us were new to CoffeeScript;
overzealous extrapolation from Python syntax resulting
in many unpleasant debugging experiences.

We opted to use fewer ``out-of-the-box'' \texttt{d3.js} solutions such as
the force-directed graph, so we weren't able to provide more advanced features
like bouncing our graph around and stretching edges.
However, we realized that it was more important to develop a working product,
so we remained with our simpler representations.

\subsection{Work distribution}

The two authors distributed the work equally.
Tony primarily worked on the pebble algorithm and infinitesimal rigidity implementations,
while Menghua designed and created the GUI.
Both contributed equally to the writing of this paper.

\subsection{Code}

A live demonstration of this application may be found at
\url{http://web.mit.edu/rmwu/www/rigid/} for the time being.
Rigiditea's source code is located in a public git repository,
\url{https://github.com/rmwu/rigiditea}.

The project was named after bubble tea since we really like bubble tea.
For the interested reader, we used many \texttt{console.log} statements
with different flavors of bubble tea to quickly track errors.
Many remain for the entertainment of confused and/or curious users.

\begin{acknowledgments}
We would like to thank Professor Erik Demaine for his instruction,
humor, and guidance. We would also like to thank the 6.849 course staff,
Martin Demaine, Dr. Jason Ku, Adam Hesterberg, and Jayson Lynch
for their continual support throughout the semester.
Finally, we would like to thank Ludwig van Beethoven for providing
music in the wee hours of the morning.
\end{acknowledgments}

\bibliography{rigiditea}{}
\bibliographystyle{plain}






\end{document}

