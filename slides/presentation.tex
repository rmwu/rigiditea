% TODO change date

%\documentclass[handout,notes]{beamer}
\documentclass[]{beamer}
\usetheme{CambridgeUS}
\usecolortheme{dolphin}
\usepackage{tikz}

\input{6up.tex}
\newcommand{\vocab}[1]{\textcolor{blue}{\textbf{#1}}}
\newcommand{\abs}[1]{|#1|}


%\usepackage{beamerthemesplit} % Activate for custom appearance

\title{Rigiditea}
\author{Rachel Wu, Tony Zhang}
\date{May 10, 2017}

\begin{document}

\frame{\titlepage}

\section[Outline]{}
\frame{\tableofcontents}

\frame{
    \frametitle{Linkage rigidity suite}
    	Our project explores rigidity in 2D linkages
	through a GUI-based web application.
	In particular, we
	
	\begin{enumerate}
	    \item created an interface for designing and manipulating linkages,
	    \item implemented and visualized iterations of the Pebble algorithm,
	    \item and implemented an infinitesimal rigidity tester.
	\end{enumerate}
}

\frame{
    \frametitle{Design goals}
    	In building a web application, we had the following goals in mind.
	
	\begin{enumerate}
	    \item Web interface should be easy to use and understand,
	    especially for users unfamiliar with linkages.
	    \item Algorithms should be visualized in an ``intuitive'' way,
	    to facilitate their understanding.
	    \item Try out CoffeeScript! (and polish up on \texttt{d3.js}?)
	\end{enumerate}
}

\section{Interface}

\frame{
    \frametitle{GUI interface}
    Our user interface has several features. Users are currently able to
    
    \begin{enumerate}
    	\item design graphs by clicking and dragging, and
	\item step through the Pebble algorithm,
	one iteration at a time.
    \end{enumerate}
    \begin{center}
    	\includegraphics[scale=.2]{fig/fig-gui-basic}
    \end{center}

    In the near future, we will add infinitesimal rigidity visualization,
    whose backend has already been written.
}

\frame{
    \frametitle{Underlying linkage representation}
    \begin{itemize}
    \item We represent linkages as \texttt{PebbleGraph} objects,
    which are undirected graphs specified with a set of vertices
    and a set of edges.
    \item Each graph has vertices and edges,
    which carry their own attributes (e.g. number of pebbles)
    \end{itemize}
}


\section{The pebble game}

\frame{
    \frametitle{Preliminaries}
    \begin{itemize}
    \item
    We'd like to determine the generic rigidity
    of undirected graph $G = (V, E)$.
    \item
    Let $n = \abs{V}$ and $m = \abs{E}$.
    \end{itemize}
}

\frame{
    \frametitle{Independence}
    \begin{itemize}
    \item<1->
    A subset of the edges in the graph is \vocab{independent}
    if, in a generic realization of the graph,
    the constraints imposed by those edges (rows in the rigidity matrix)
    are linearly independent.
    \item<2->
    A \vocab{maximal independent set} is one that cannot be enlarged while preserving independence.
    \end{itemize}
}


\frame{
    \frametitle{Laman rigidity}
    \begin{itemize}
    \item<1->
    $n$ free vertices have $2n-3$ degrees of freedom in the plane.
    (Two per vertex, three dimensional isometry space.)
    \item<2->
    Laman:
    the edges of graph $G$ are independent in two dimensions
    if no subgraph induced by $n'$ vertices
    has more than $2n' - 3$ edges.
    \item<3->
    Corollary: if $G$ has $2n-3$ edges
    and no induced subgraph has more than $2n'-3$,
    $G$ is minimally generically rigid.
    \item<4->
    Corollary: $G$ is generically rigid if it has $2n-3$ independent edges.
    \end{itemize}
}

\frame{
    \frametitle{The game}
    \begin{itemize}
        \item<1->
        Given undirected graph $G = (V, E)$,
        give two pebbles to each vertex.
        Each vertex can use its pebbles to cover incident edges.
        \item<2->
        A \vocab{pebble covering} is an assignment of pebbles
        such that each edge is covered by a pebble.
        \item<3->
        Theorem: Given independent edge set $\hat E$,
        another edge $e$ is independent iff there exists a pebble covering
        of the graph $G_{4e}$ consisting of $\hat E$ and four copies of $e$.
    \end{itemize}
}

\frame{
    \frametitle{The algorithm}
    \begin{itemize}
    \item<1->
    To test rigidity, we want to check whether maximal independent edge set has $2n-3$ elements.
    \item<2->
    We build up an independent edge set $\hat E$
    by greedily adding edges if they are independent
    (tested by playing the pebble game on $G_{4e}$).
    \end{itemize}
}

\section{Extensions}

\begin{frame}
    \frametitle{Work yet to be completed}
    There are several items we plan to complete.
    \begin{enumerate}
        \item Smooth out the GUI with animations
        and possibly, user-friendly keyboard shortcuts.
        \item Introduce more granularity in each ``step'' of the Pebble algorithm.
        \item Animate or visualize infinitesimal rigidity results on the graph as well.
    \end{enumerate}
    In general, we plan to improve the GUI for a user-friendly and intuitive experience.
\end{frame}



\begin{frame}
  \frametitle{Further Reading}
  \begin{thebibliography}{10}
  \beamertemplatebookbibitems
  \bibitem{pebble}
  Donald J. Jacobs, Bruce Hendrickson, An Algorithm for Two-Dimensional Rigidity Percolation: The Pebble Game, \textit{J.~Comp.~Phys.} 137, 346-365 (1997)
  \end{thebibliography}
\end{frame}

\end{document}
