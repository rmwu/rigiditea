% TODO change date

%\documentclass[handout,notes]{beamer}
\documentclass[]{beamer}
\usetheme{Singapore}
\usecolortheme{dolphin}
\usepackage{tikz}

\input{6up.tex}
\newcommand{\vocab}[1]{\textcolor{blue}{\textbf{#1}}}

%\usepackage{beamerthemesplit} % Activate for custom appearance

\title{Rigiditea}
\author{Rachel Wu, Tony Zhang}
\date{May 10, 2017}

\begin{document}

\frame{\titlepage}

\section[Outline]{}
\frame{\tableofcontents}

\frame{
  \frametitle{Caveat emptor}
  \begin{itemize}
  \item<1-> I will say some true things. I will also say some false things.
  \item<2-> There will be math.
  \item<3-> There will be handwaving.
  \end{itemize}
}


\section{Interface}





\section{The pebble game}

\frame{
    \frametitle{Conditions for rigidity}
    
}

\frame{
    \frametitle{The game}
    \begin{itemize}
        \item
        Given undirected graph $G = (V, E)$,
        give two pebbles to each vertex.
        Each vertex can use its pebbles to cover incident edges.
        \item
        A \vocab{pebble covering} is an assignment of pebbles
        such that each edge is covered by a pebble.
        \item
        
    \end{itemize}
}

\section{Extensions}





\begin{frame}
  \frametitle{Further Reading}
  \begin{thebibliography}{10}
  \beamertemplatebookbibitems
  \bibitem{pebble}
  Donald J. Jacobs, Bruce Hendrickson, An Algorithm for Two-Dimensional Rigidity Percolation: The Pebble Game, \textit{J.~Comp.~Phys.} 137, 346-365 (1997)
  \end{thebibliography}
\end{frame}

\end{document}
